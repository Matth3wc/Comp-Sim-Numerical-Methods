\documentclass[12pt,a4paper]{article}
\usepackage[margin=2.5cm]{geometry}
\usepackage{amsmath,amssymb}
\usepackage{graphicx}
\usepackage{booktabs}
\usepackage{siunitx}
\usepackage{physics}
\usepackage{hyperref}
\usepackage{caption}
\usepackage{subcaption}
\usepackage{float}

\sisetup{
    separate-uncertainty=true,
    multi-part-units=single
}

\title{Measurement of Magnetic Susceptibility and Saturation Magnetisation Using the Faraday Method}
\author{Matthew Conway\\Student ID: 22337426\\Trinity College Dublin}
\date{February 2026}

\begin{document}

\maketitle

\begin{abstract}
The magnetic susceptibility of gadolinium gallium garnet (GGG) and the saturation magnetisation of haematite ($\alpha$-Fe$_2$O$_3$) were measured using the Faraday method with an electromagnet capable of producing fields up to \SI{1}{\tesla}. The apparatus was calibrated using Mohr's salt, yielding a field gradient constant $C = \SI{7}{\per\metre}$. For GGG, the measured mass susceptibility was $\chi_m = \SI{8.2(14)e-7}{\metre\cubed\per\kilogram}$, in good agreement with the Curie law prediction of $\SI{1.0e-6}{\metre\cubed\per\kilogram}$ (ratio 0.8). For haematite, the saturation magnetisation was $\sigma_s = \SI{1.07}{\ampere\metre\squared\per\kilogram}$, corresponding to \SI{0.03}{\mu_B} per Fe$_2$O$_3$ formula unit—consistent with haematite's canted antiferromagnetic structure. These results demonstrate the validity of the Faraday method for characterising both paramagnetic and weakly ferromagnetic materials.
\end{abstract}

\section{Introduction}

The magnetic properties of materials are of fundamental importance in condensed matter physics and have widespread technological applications. This experiment employs the Faraday method to measure two key magnetic quantities: the mass susceptibility $\chi_m$ of a paramagnetic material and the saturation magnetisation $\sigma_s$ of a ferromagnetic material.

\subsection{Theoretical Background}

\subsubsection{Force on a Magnetic Sample}

In an inhomogeneous magnetic field, a magnetised sample experiences a translational force. For a sample with magnetic moment $\mathbf{m}$ in a field with gradient, the force is:
\begin{equation}
    F_x = m \cdot \frac{\partial B_x}{\partial x}
\end{equation}

For a paramagnetic material, the induced magnetisation is proportional to the applied field:
\begin{equation}
    M = \chi_m \rho B_x / \mu_0
\end{equation}
where $\chi_m$ is the mass susceptibility (\si{\metre\cubed\per\kilogram}), $\rho$ is the density, and $\mu_0 = \SI{1.257e-6}{\tesla\metre\per\ampere}$ is the permeability of free space.

The force on a paramagnetic sample of mass $m$ becomes:
\begin{equation}
    F = \frac{m \chi_m}{\mu_0} B_x \frac{\partial B_x}{\partial x} = \frac{m \chi_m}{2\mu_0} \frac{\partial B_x^2}{\partial x}
\end{equation}

Defining the field gradient constant $C \equiv \frac{1}{2\mu_0}\frac{\partial B_x^2}{\partial x} / B_x^2$, the apparent weight change is:
\begin{equation}
    \Delta m \cdot g = m \chi_m C B^2
    \label{eq:paramagnet}
\end{equation}

For a ferromagnetic material with saturation magnetisation $\sigma_s$ (\si{\ampere\metre\squared\per\kilogram}):
\begin{equation}
    \Delta m \cdot g = m \sigma_s C B
    \label{eq:ferromagnet}
\end{equation}

\subsubsection{Curie Law for Paramagnets}

For a paramagnetic material containing magnetic ions with total angular momentum quantum number $J$ and Landé g-factor $g_J$, the Curie law predicts:
\begin{equation}
    \chi_m = \frac{\mu_0 N g_J^2 \mu_B^2 J(J+1)}{3 k_B T}
    \label{eq:curie}
\end{equation}
where $N$ is the number of magnetic ions per unit mass, $\mu_B = \SI{9.274e-24}{\joule\per\tesla}$ is the Bohr magneton, and $k_B = \SI{1.381e-23}{\joule\per\kelvin}$ is Boltzmann's constant.

\subsubsection{Antiferromagnetism in Haematite}

Haematite ($\alpha$-Fe$_2$O$_3$) is not a simple ferromagnet but rather a \emph{canted antiferromagnet}. Below the Néel temperature ($T_N = \SI{948}{\kelvin}$), the Fe$^{3+}$ spins align antiparallel, but with a slight canting angle of approximately \SI{0.1}{\degree}. This results in a weak net ferromagnetic moment, typically \SI{0.01}{}-\SI{0.1}{\mu_B} per formula unit, rather than the \SI{10}{\mu_B} expected for aligned Fe$^{3+}$ ions (each with $S = 5/2$).

\section{Experimental Method}

\subsection{Apparatus}

The experiment used an electromagnet with tapered pole pieces to produce an inhomogeneous field up to \SI{1}{\tesla}. The sample was suspended from an analytical balance (resolution \SI{0.1}{\milli\gram}) positioned above the magnet gap. The magnetic field was measured using a Hall probe, and the current was controlled via a variable power supply.

\subsection{Calibration}

The field gradient constant $C$ was determined using Mohr's salt ((NH$_4$)$_2$Fe(SO$_4$)$_2 \cdot$6H$_2$O), which has a well-characterised susceptibility of $\chi_m = \SI{0.33}{\joule\per\tesla\squared\per\kilogram}$ at room temperature. From Equation~\ref{eq:paramagnet}:
\begin{equation}
    C = \frac{\text{slope} \times g}{m_{\text{Mohr}} \times \chi_{\text{Mohr}}}
\end{equation}
where the slope is obtained from a linear fit of $\Delta m$ versus $B^2$.

\subsection{Samples}

Three samples were studied:
\begin{itemize}
    \item \textbf{Mohr's salt}: $m = \SI{73.9}{\milli\gram}$, $\chi_m = \SI{0.33}{\joule\per\tesla\squared\per\kilogram}$ (calibration standard)
    \item \textbf{GGG} (Gd$_3$Ga$_5$O$_{12}$): $m = \SI{24.4}{\milli\gram}$, paramagnetic garnet with Gd$^{3+}$ ions ($J = 7/2$, $g_J = 2$)
    \item \textbf{Haematite} ($\alpha$-Fe$_2$O$_3$): $m = \SI{22.2}{\milli\gram}$, canted antiferromagnet
\end{itemize}

\subsection{Measurement Procedure}

\begin{enumerate}
    \item The magnetic field calibration $B(I)$ was established using the Hall probe.
    \item Background measurements were taken with an empty sample holder to correct for holder diamagnetism and field-dependent buoyancy effects.
    \item For each sample, the apparent weight was recorded as the current was ramped from 0 to maximum (forward sweep) and back to 0 (reverse sweep).
    \item Hysteresis between forward and reverse sweeps was monitored to assess systematic effects.
\end{enumerate}

\section{Results}

\subsection{Field Calibration}

The magnetic field varied linearly with current over the range studied, with $\mathrm{d}B/\mathrm{d}I = \SI{0.19}{\tesla\per\ampere}$.

\subsection{Calibration Constant}

Figure~\ref{fig:mohrs} shows the apparent weight change versus $B^2$ for Mohr's salt. A linear fit yields:
\begin{equation}
    \frac{\mathrm{d}(\Delta m)}{\mathrm{d}(B^2)} = \SI{17.5}{\milli\gram\per\tesla\squared}
\end{equation}

The field gradient constant is therefore:
\begin{equation}
    C = \frac{\SI{17.5e-6}{\kilogram\per\tesla\squared} \times \SI{9.81}{\metre\per\second\squared}}{\SI{73.9e-6}{\kilogram} \times \SI{0.33}{\joule\per\tesla\squared\per\kilogram}} = \SI{7}{\per\metre}
\end{equation}

\begin{figure}[H]
    \centering
    \includegraphics[width=0.7\textwidth]{plots/exp1_mohrs.png}
    \caption{Apparent weight change versus $B^2$ for Mohr's salt calibration standard. The linear fit (green) gives a slope of \SI{17.5}{\milli\gram\per\tesla\squared}.}
    \label{fig:mohrs}
\end{figure}

\subsection{Paramagnetic Susceptibility of GGG}

Figure~\ref{fig:ggg} shows results for gadolinium gallium garnet. The response is linear with $B^2$ for $B < \SI{0.9}{\tesla}$, as expected for a paramagnet. At higher fields, deviation from linearity indicates approach to saturation.

Fitting the linear region yields:
\begin{equation}
    \frac{\mathrm{d}(\Delta m)}{\mathrm{d}(B^2)} = \SI{11.4}{\milli\gram\per\tesla\squared}
\end{equation}

The experimental susceptibility in laboratory units is:
\begin{equation}
    \chi_{m,\text{lab}} = \frac{\SI{11.4e-6}{\kilogram\per\tesla\squared} \times \SI{9.81}{\metre\per\second\squared}}{\SI{24.4e-6}{\kilogram} \times \SI{7}{\per\metre}} = \SI{0.65}{\joule\per\tesla\squared\per\kilogram}
\end{equation}

Converting to SI units ($\chi_{\text{SI}} = \chi_{\text{lab}} \times \mu_0$):
\begin{equation}
    \chi_m = \SI{8.2(14)e-7}{\metre\cubed\per\kilogram}
\end{equation}

\subsubsection{Comparison with Curie Law}

For GGG, the Gd$^{3+}$ ion has $J = 7/2$ and $g_J = 2$. The molar mass is $M = \SI{872.1}{\gram\per\mole}$, containing 3 Gd$^{3+}$ ions per formula unit. At $T = \SI{293}{\kelvin}$:
\begin{equation}
    N = \frac{3 N_A}{M} = \SI{2.07e24}{\per\kilogram}
\end{equation}

The Curie law prediction (Equation~\ref{eq:curie}) gives:
\begin{equation}
    \chi_{m,\text{Curie}} = \SI{1.00e-6}{\metre\cubed\per\kilogram}
\end{equation}

The ratio $\chi_{m,\text{exp}}/\chi_{m,\text{Curie}} = 0.8$, indicating good agreement within experimental uncertainty.

\begin{figure}[H]
    \centering
    \includegraphics[width=0.7\textwidth]{plots/exp2_ggg.png}
    \caption{Apparent weight change versus $B^2$ for GGG. The linear fit (green) gives a slope of \SI{11.4}{\milli\gram\per\tesla\squared}. The shaded region indicates high-field saturation effects.}
    \label{fig:ggg}
\end{figure}

\subsection{Saturation Magnetisation of Haematite}

Figure~\ref{fig:haematite} shows the hysteresis loop for haematite, plotting $\sigma(B)$ calculated from Equation~\ref{eq:ferromagnet}. The characteristic S-shaped curve and hysteresis confirm ferromagnetic behaviour.

From high-field data ($|B| > \SI{0.5}{\tesla}$), linear extrapolation to $B = 0$ gives the saturation magnetisation:
\begin{equation}
    \sigma_s = \SI{1.07}{\ampere\metre\squared\per\kilogram}
\end{equation}

\subsubsection{Magnetic Moment per Formula Unit}

The moment per Fe$_2$O$_3$ formula unit is:
\begin{equation}
    n_{\mu_B} = \frac{\sigma_s M}{N_A \mu_B} = \frac{\SI{1.07}{\ampere\metre\squared\per\kilogram} \times \SI{159.7e-3}{\kilogram\per\mole}}{\SI{6.022e23}{\per\mole} \times \SI{9.274e-24}{\ampere\metre\squared}} = 0.031\,\mu_B
\end{equation}

This is only 0.3\% of the theoretical maximum of \SI{10}{\mu_B} for aligned Fe$^{3+}$ ions, consistent with haematite's canted antiferromagnetic structure.

\begin{figure}[H]
    \centering
    \includegraphics[width=0.7\textwidth]{plots/exp3_haematite.png}
    \caption{Hysteresis loop for haematite showing specific magnetisation $\sigma$ versus applied field $B$. The weak ferromagnetic moment ($\sigma_s \approx \SI{1}{\ampere\metre\squared\per\kilogram}$) arises from spin canting in the antiferromagnetic structure.}
    \label{fig:haematite}
\end{figure}

\section{Discussion}

\subsection{Validity of Results}

The experimental results are physically reasonable:

\begin{enumerate}
    \item \textbf{GGG susceptibility}: The measured $\chi_m = \SI{8.2e-7}{\metre\cubed\per\kilogram}$ is 80\% of the Curie law prediction. This slight reduction is expected due to:
    \begin{itemize}
        \item Antiferromagnetic exchange interactions between Gd$^{3+}$ ions
        \item Crystal field effects modifying the effective moment
        \item Temperature uncertainty (actual sample temperature may exceed \SI{293}{\kelvin})
    \end{itemize}
    
    \item \textbf{Haematite magnetisation}: The moment of \SI{0.03}{\mu_B} per Fe$_2$O$_3$ agrees well with literature values for the canted antiferromagnetic phase (\SIrange{0.01}{0.1}{\mu_B}). The small moment arises from a spin canting angle of approximately:
    \begin{equation}
        \theta \approx \arcsin\left(\frac{0.03}{10}\right) \approx \SI{0.17}{\degree}
    \end{equation}
\end{enumerate}

\subsection{Limitations and Sources of Error}

\subsubsection{Systematic Uncertainties}

\begin{enumerate}
    \item \textbf{Calibration standard}: The susceptibility of Mohr's salt ($\chi = \SI{0.33}{\joule\per\tesla\squared\per\kilogram}$) is taken from the laboratory manual without independent verification. Any error in this value propagates directly to all subsequent measurements.
    
    \item \textbf{Unit conversion}: The laboratory susceptibility unit (\si{\joule\per\tesla\squared\per\kilogram}) differs from SI (\si{\metre\cubed\per\kilogram}) by a factor of $\mu_0$. Care must be taken to apply this conversion consistently when comparing with theoretical predictions.
    
    \item \textbf{Field gradient uniformity}: The constant $C$ assumes a specific field profile. Sample positioning affects the effective gradient experienced.
    
    \item \textbf{Temperature}: All measurements were performed at ambient temperature, assumed to be \SI{293}{\kelvin}. The Curie law susceptibility scales as $1/T$, so a \SI{5}{\kelvin} uncertainty introduces a 2\% error.
\end{enumerate}

\subsubsection{Random Uncertainties}

\begin{enumerate}
    \item \textbf{Balance precision}: The \SI{0.1}{\milli\gram} resolution limits precision, particularly at low fields where $\Delta m < \SI{1}{\milli\gram}$.
    
    \item \textbf{Hysteresis}: Differences between forward and reverse sweeps (visible in Figures~\ref{fig:mohrs}--\ref{fig:ggg}) indicate magnetic history effects in the electromagnet.
    
    \item \textbf{Fitting uncertainty}: Linear regression of $\Delta m$ vs $B^2$ yields slope uncertainties of approximately 10--15\%.
\end{enumerate}

\subsubsection{Physical Limitations}

\begin{enumerate}
    \item \textbf{Saturation effects}: For GGG at high fields ($B > \SI{0.9}{\tesla}$), the response deviates from $\Delta m \propto B^2$, indicating approach to magnetic saturation where the Curie law breaks down.
    
    \item \textbf{Demagnetising fields}: For haematite, internal demagnetising fields may reduce the effective field experienced by the sample.
    
    \item \textbf{Sample purity}: Impurities or structural defects in the samples could affect measured magnetic properties.
\end{enumerate}

\subsection{Suggestions for Improvement}

\begin{enumerate}
    \item Use a temperature-controlled sample environment to enable variable-temperature measurements and verify Curie law scaling.
    \item Employ a SQUID magnetometer for higher sensitivity at low fields.
    \item Measure multiple calibration standards to cross-check the gradient constant.
    \item Use smaller field increments to better resolve hysteresis and saturation effects.
\end{enumerate}

\section{Conclusion}

The Faraday method has been successfully employed to characterise the magnetic properties of two materials:

\begin{enumerate}
    \item \textbf{Gadolinium gallium garnet}: The measured mass susceptibility $\chi_m = \SI{8.2(14)e-7}{\metre\cubed\per\kilogram}$ agrees with the Curie law prediction to within 20\%, validating the paramagnetic model for this rare-earth garnet.
    
    \item \textbf{Haematite}: The saturation magnetisation $\sigma_s = \SI{1.07}{\ampere\metre\squared\per\kilogram}$ corresponds to a moment of \SI{0.03}{\mu_B} per formula unit, consistent with haematite's canted antiferromagnetic structure rather than true ferromagnetism.
\end{enumerate}

These results demonstrate that the Faraday balance technique, while relatively simple, provides quantitatively accurate measurements of magnetic susceptibility and magnetisation when proper calibration and unit conversions are applied.

\section*{References}

\begin{enumerate}
    \item Cullity, B.D. and Graham, C.D., \textit{Introduction to Magnetic Materials}, 2nd ed., Wiley (2009).
    \item Morrish, A.H., \textit{Canted Antiferromagnetism: Hematite}, World Scientific (1994).
    \item Laboratory Manual, ``Magnetisation and Susceptibility'', Trinity College Dublin.
\end{enumerate}

\end{document}
